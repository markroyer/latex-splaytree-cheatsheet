\documentclass{memoir}

\usepackage{calc}

\usepackage{tikz}
\usepackage{graphicx}
\usepackage{lstcustom}

\usetikzlibrary{shapes.geometric}

\usepackage[margin=0.5in]{geometry}

\usepackage{float}
\floatstyle{boxed}
\restylefloat{figure}





\tikzset{
  treenode/.style = {align=center, inner sep=0pt, text centered,
    font=\sffamily},
  tnode/.style = {treenode, circle, black, draw=black, inner sep=2pt, minimum width=2em},
  subtree/.style = {draw,dashed,shape border uses incircle,
  isosceles triangle,shape border rotate=90, minimum height=1cm},
  tnull/.style = {treenode, rectangle, draw=black,
    minimum width=0.5em, minimum height=0.5em}
}



\newcommand{\treespace}{\vspace{12pt}}

\newcommand{\treewidth}{.6\textwidth}




\begin{document}

\begin{vplace}

\begin{figure}[h!]
\begin{center}

 \textbf{\huge Splay Tree Rotations} 

\treespace

\resizebox{\treewidth}{!}{%
\begin{tikzpicture}[level/.style={sibling distance = 2cm), level distance=2cm}]


% Start initial tree
\node [tnode] (y) {y}
      [child anchor=north] 
   child{ node [tnode] {x}
      [child anchor=north]
      child{ node [subtree] {A} }
      child{ node [subtree] {B} }
    }
    child{
       node [subtree] (c) {C} 
    }       
; 
% End initial tree

% Start initial tree
\node [tnode, right of=y, right=5cm] (xnew) {x}
    [child anchor=north] 
    child{
       node [subtree] (anew) {A} 
    } 
   child{ node [tnode] (ynew) {y}
      [child anchor=north]
      child{ node [subtree] {B} }
      child{ node [subtree] {C} }
    }

; 

\draw[->, shorten <=.75cm, shorten >=.75cm][thick] (c) to[out=0,in=180] node [auto]{Zig} (anew);

%\draw[gray,step=0.25] (-2.5,-4.5) grid (9, .5);

\end{tikzpicture}
}%

\treespace

\resizebox{\treewidth}{!}{%
  \begin{tikzpicture}[level/.style={sibling distance = 2cm), level distance=2cm}] %[level/.style={sibling distance = 5cm/(2^#1), level distance=1.5cm}] 


% Start initial tree
\node [tnode] (z) {z}
      [child anchor=north] 
   child{ node [tnode] {y}
      [child anchor=north]
      child{
        node [tnode] {x}
        child { node [subtree] {A} }
        child { node [subtree] {B} }
      }
      child{ node [subtree] {C} }
    }
    child{
       node [subtree] (d) {D} 
    }       
; 
% End initial tree

% Start initial tree
\node [tnode, right of=y, right=5cm] (xnew) {x}
    [child anchor=north] 
    child{
       node [subtree] (anew) {A} 
    } 
    child {
      node [tnode] {y}
      child { node [subtree] {B} }
      child { node [tnode] (znew) {z}
        [child anchor=north]
        child{ node [subtree] {C} }
        child{ node [subtree] {D} }
      }
    }

; 

\draw[->, shorten <=.75cm, shorten >=.75cm][thick] (d) to[out=0,in=180] node [auto]{Zig-zig} (anew);

\end{tikzpicture}
}%

\treespace

\resizebox{\treewidth}{!}{%
\begin{tikzpicture}[level/.style={sibling distance = 2cm), level distance=2cm}]


% Start initial tree
\node [tnode] (z) {z}
      [child anchor=north] 
   child{ node [tnode] {y}
      [child anchor=north]
      child{ node [subtree] {A} }
      child{
        node [tnode] {x}
        child { node [subtree] {B} }
        child { node [subtree] {C} }
      }
    }
    child{
       node [subtree] (d) {D} 
    }       
; 
% End initial tree

% Start initial tree
\node [tnode, right of=y, right=6cm] (xnew) {x}
    [child anchor=north, level/.style={sibling distance=6cm/2^#1}] 
    child{
      node [tnode] (ynew) {y} 
      child { node [subtree] {A} }
      child { node [subtree] {B} }
    } 
    child {
      node [tnode] {z}
      child { node [subtree] {C} }
      child { node [subtree] {D} }
    }

; 

\draw[->, shorten <=.75cm, shorten >=.75cm][thick] (d) to[out=0,in=180] node [auto]{Zig-zag} (anew);

\end{tikzpicture}
}%

The figure does not depict symmetric
 cases.

\end{center}
\end{figure}


\section*{When to splay}

\begin{itemize}

\item \lstinline$search(element)$: If the element is found in a node \textit{u}, we splay
  \textit{u}.  Otherwise, we splay the node where the search
  terminates unsuccessfully.

\item \lstinline$insert(element)$: We splay the newly created node that contains the element.

\item \lstinline$remove(element)$: We splay the parent of the node that contains the element. If
  the node is the root, we splay its left child or right child. If the
  element is not in the tree, we splay the node where the search
  terminates unsuccessfully.

\end{itemize}


\end{vplace}


\end{document}
